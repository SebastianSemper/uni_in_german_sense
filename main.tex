\documentclass[a4paper,10pt,twoside,titlepage]{article}
% 
\usepackage[utf8]{inputenc}
\usepackage{fontenc}
\usepackage[]{hyperref}
\usepackage{fullpage}

\usepackage{multicol}
\usepackage{color}
 
\usepackage{comment}

\setlength{\columnseprule}{1pt}
\def\columnseprulecolor{\color{blue}}

\author{Cordelia Krauß, Max Lörtzing, Sebastian Semper}
\title{Gelegentliche Gedanken über die Insignifikanz der Universitäten im Deutschen Sinne}
\date{2015-04-06}

\begin{document}



\maketitle

\tableofcontents

\newpage


\section{Einleitung}
selbst Student, interesse an guter bildung weil bildung den geist befreit, idealist weil am eigenen leib erfahren wie fähigkeit zur selbstständigen wissensbeschaffung aussicht auf schönes leben bereitet

%unmut, unzufriedenheit, hilflosigkeit, selbstreflexion
\subsection{Motivation}
bildung wird zu oft schön geredet und an den falschen stellen schlecht. oft wird fehler bei studenten nicht den besuchten einrichtungen gesucht. junger mensch verbringt viel zeit in schule, ergo schule formt ihn sehr. unzufriedenheit, wie leicht ersichtliche probleme ignoriert werden, aus hilfosigkeit gegenüber politik. selbst nur fähig missstände zu sichten, nicht an behebung zu arbeiten

%beobachtungen, eigene erfahrungen als student
\subsection{Ausgangssituation}
studenten an TU/FH einmal den ganzen kreislauf miterlebt, erfahrungen als student, tutor, seminarleiter wissenschaftlicher mitarbeiter, nachhilfelehrer. nicht viel mitbekommen, aber einige umstände sind stellvertretend für gesamtsituation.

\section{Missqualifikation der Bewerber}
universitäten per se nicht selbst schuld an situation. schulen erzeugen defizite und stellen zu viele hochschulreifen aus. hochschule als nachfolge anstalt der schule kann ohne vorherige analyse der schule nicht verstanden werden.

%immer mehr bekommen eine unverdient
\subsection{Abwertung der Hochschulreife}
fallbeispiele: kein th in englisch aussprechen können, keine quadrat gleichungen lösen können, keinen text auf wesentliche inhalte untersuchen diskussion kritik des inhalts, keinen text flüssig laut vorlesen können, keine prozentrechnung, keine bedienung eines katalogs,

abwertung der hsr, da viele eine bekommen ohne ansatzweise fähig zu sein an einer uni zu studieren. (Mangel an geistigen Fähigkeiten (Logik,Methodik) und mangelndes Handwerkszeug(siehe oben)  

%ausbildung von arbeitern statt menschen
\subsection{Ökonomie statt Humanismus}
immer mehr allgemeine prinzipien, wie musik, kunst, philosophie, geschichte fallen technisch und praxis orientierten fächern zum opfer. keine lernt mehr altgriechisch/latein, da es als sinnlos gesehen wird. natürlich von schülern, aber auch immer mehr von erwachsenen/pädagogen. dadurch weniger beschäftigung mit unseren kulturellen wurzeln, die in rom/athen lagen. weniger musik/kunst-verständnis, weniger sinn für dinge, die nicht per se wirtschaftlich verwertbar sind, sondern dem mensch als ganzes abwechslung und möglichkeiten des genusses bringen. mehr genuss darin schönen code oder effizient gearbeitet zu haben, als ein schönes lied zu hören und zu verstehen. bildung als weg in die arbeitswelt anstatt weg zur eigenen identität.

Naturwissenschafften werden zwar bewundert jedoch nur als Wegbereiter der technisch-praktischen Studiengänge. Das humboltsche Bildungsideal wurde kritisiert weil es zu humanistisch und zu wenig technisch war. Anstatt dieses anzugleichen wurde es nahezu komplett ersetzt. (Bologna)


%keine denkweisen und ideen, nur noch fakten
\subsection{Vernachlässigung der Methodik}
deutschaufsätze als schemata nicht als schöpferische werke, analysen als kochrezept und nicht als auch vom gefühl und empfinden beeinflusste arbeiten, mathematik als bloße abarbeitung von algorithmen ohne hintergrund, fremdsprachen als mittel der kommunikation nicht als weg zur eigenen kulturellen wurzel. vordringen in fremde gedanken, sei es durch geschichten, gedichte oder sachtexte wird vernachlässigt. fakten werden aufbereitet und das meist simplifiziert und ohne zusammenhang dargestellt. 

\section{Hilflosigkeit der Hochschule}
hochschule ist also eher ein sammelplatz der schon von vorn herein nicht wirklich auf uni vorbereiteten leute. uni muss damit zurechtkommen. muss sie das? was für folgen gibt es? für uni, für studenten? hochschulreife exisitert nur auf papier. studieren auch ohne hoschschulreife möglich->Funktioniert nur wenn das Niveau angeglichen wird. 

%ausbügeln der defizite vermeintlicher hochschulreifen
\subsection{Schmelztigel der Defizite}
mathekurse, bibliothekskurse, recherchekurse, englischkurse, es wird davon ausgegangen, dass studenten praktisch nichts selber mehr können. zeit die eigentlich für neue sachen genutzt werden sollte wird mit altem schnee bedeckt

%schwerpunkte verschieben sich
\subsection{Von der Forschung zur Bildung}
wenier ausbildung von arbeitern: lerninhalte, lernmethoden. druck aus wirtschaft? druck von den studenten selbst?

zahl der bewerber ist maßgebend für geldmittel. Allen wird versprochen studieren zu dürfen. Aber kann er auch studieren. (Interview mit Helge Kienel Kunsthistoriker und Dozent an einer Hochschule) Aufgebaute Existenzen gehen zu Grunde weil Leute Ihren Beruf aufgeben um eine angebliche Chance auf ein besseres Leben zu erhalten. Abbrecherquote gestiegen (muss noch statistisch belegt werden) Veröffentlichte Zahlen sind meistes die der Studienanfänger selten die der Absolventen. 
\subsection{Finanzielle Abhängigkeit}
Je mehr Studenten desto mehr Geld bekommt die Universität. Stichwort: Unternehmensorientiert anstatt Forschungsorientiert. 

%viele absolventen werden als erfolge gefeiert, Chancengleichheit
\subsection{Betriebsblindheit}

\section{Eventuelle Ursachen und Auswirkungen}

%frühere ausbildungsberufe werden studiengänge. Arbeitsfeld hat sich jedoch nicht gewandelt. Falls doch: nicht in die wissenschaftliche Richtung.
\subsection{Künstliche Akademisierung} 

%zu wenige absolventen an uni => absenkung des niveaus => kein positiver effekt
\subsection{Fachkräftemangel}

%mehr fachkräfte von außen
\subsection{Notwendigkeit des Bildungsimports}
aber sinkendes niveau führt zur gegenseitigen negativen katalyse beider obiger punkte
\end{document}
